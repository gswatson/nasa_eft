\documentclass[useAMS,usenatbib,a4paper,12pt]{article}
%\usepackage{fullpage}% 1 inch margins 
%\usepackage{epsfig}
\usepackage{graphicx}

%\usepackage{deluxetable}
\usepackage{amssymb}
\usepackage{mathrsfs}
\usepackage{amsmath}
\usepackage{amssymb}

%\usepackage{draftwatermark}
%\SetWatermarkScale{5}

\pagestyle{myheadings}\makeatletter
\pagenumbering{arabic}  

\usepackage{color}
\newcommand{\hilight}[1]{\colorbox{yellow}{#1}}

\usepackage[
top    = 1in,
bottom = 1in,
left   = 1in,
right  = 1in]{geometry}

%\usepackage{arial}

\newcommand{\apj}{ApJ}
\newcommand{\apjl}{ApJ Lett}
\newcommand{\apjs}{ApJS}
\newcommand{\apss}{Ap \& SS}
\newcommand{\bain}{Bull. Astron. Inst. Netherlands} 
\newcommand{\mnras}{MNRAS}
\newcommand{\aj}{AJ}
\newcommand{\aap}{A\&A}
\newcommand{\jcap}{JCAP}
\newcommand{\prd}{Phys Rev D}
\newcommand{\prl}{Phys Rev Lett}
\newcommand{\aapr}{A\&A Rev}
\newcommand{\nat}{Nature}
\newcommand{\jrasc}{JRASC}
\newcommand{\araa}{ARA\&A}
\newcommand{\ssr}{Space Science Reviews}
\newcommand{\be}{\begin{equation}}
\newcommand{\ee}{\end{equation}}
\newcommand{\bea}{\begin{eqnarray}}
\newcommand{\eea}{\end{eqnarray}}
\def\comment{\textcolor{red}}


\date{}

\begin{document}

\title{Summary}
\maketitle

Establishing the connection between the end of the inflationary epoch in the early universe and the creation of particles in the Standard Model is an important goal of modern physics. Several fundamental questions include: {\it How does inflation eventually lead to the Standard Model?}  {\it What was the temperature of the `hot Big Bang'?}. Primordial Black Holes (PBHs) can provide an important way to probe the early universe, and in particular a possible non-standard early phase. If PBHs are created over an extended time interval, then we expect a range of PBH masses. Though there is no preferred mass scale for PBHs from theory, they may be cosmologically-significant and can be a viable candidate for the dark matter in the universe. It is theoretically well-know that PBHs can decay to Standard Model particles via Hawking radiation, providing a method to search for them in modern cosmological data sets. 

In this project we will utilize the most updated data set from the Planck satellite and modify the cosmological recombination history to account for energy injection from a spectrum of evaporating PBHs. This will improve upon existing constraints which rely on pre-Planck data. In addition we will  utilize the most updated data set from the Fermi-LAT, and reanalyze data from the Comptel gamma-ray satellite, in order to account for energy injection from a spectrum of evaporating PBHs. As is the case with our proposed CMB analysis, we anticipate that our results will improve existing constraints which typically assume that all PBHs are formed at a single mass. In addition to analyzing the extragalactic gamma-ray background, we will perform the first study of PBHs using diffuse Galactic radiation measured by Fermi-LAT.  


\end{document}

